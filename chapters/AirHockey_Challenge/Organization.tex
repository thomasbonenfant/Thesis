\section{Challenge organization}
\label{sec:organization}
The challenge consists in three simulated stages: Warm Up, Qualifying and Tournament. Each stage consists of different tasks
required for robot air hockey. Apart from executing the given tasks the robots needs to be robust and capable of adapting to changes.
Except from the Warm Up stage the submitted agents are evaluated in modified simulation environments to mimic the sim-to-real gap.

The first three teams are able to deploy their approach on the real robot and compete in a real world scenario.

\subsection{Warm-Up Stage}
In the Warm-Up stage the participants were given an ideal environment with a 3-degree-of-freedom robot. This stage aimed to familiarize with the given tasks,
the environment and the application programming interface.

The tasks given at this point were two:
\begin{itemize}
    \item \textbf{Hit}: The puck is randomly initialized on the left side of the table. The initial velocity is zero. The objective is to hit the puck
    to score a goal as fast as possible.
    \item \textbf{Defend}: The puck is randomly initialized on the right side of the table with a random velocity heading left.
    The objective is to stop the puck on the right side of the table and prevent it from getting scored.
\end{itemize}
\subsection{Qualifying Stage}
\label{subsec:qualifying_stage}
In this stage the participants were given an environment exposing an interface to control a general-purpose robot, the KUKA-iiwa14 LBR.
At this point the submitted agents to the challenge's cloud evaluator were simulated with a modified version of the environment handed to the participant teams
in order to simulate different types of real world problems. This modifications included disturbances, observation noise, loss of tracking, model mismatch, and imperfect trakcing controller.

Each team could evaluate its solution once per task per day. Each evaluation was conducted with 1000 episodes, equivalently 2.8 hours of real-world experiments.
Based on the evaluation metric, the agents were categorized into three levels:
\begin{itemize}
    \item Deployable
    \item Improvable
    \item Nondeployable.
\end{itemize}

The given tasks at this stage were three:

\begin{itemize}
    \item \textbf{Hit}: The opponent moves in a predictable pattern. The puck is randomly initialized with a small velocity. The objective is to score the goal as many times as possible.
    The episode ends when the puck is scored or it bounces back. The episodes terminates with \textit{success} if the puck is scored.
    \item \textbf{Defend}: The puck is randomly initialized on the right side of the table with a random velocity heading left.
    The objective is to stop the puck on the right side of the table and prevent it from getting scored.
    The episode terminates if the puck is returned to the opponent's side or scored or the puck speed dropbs below the threshold.
    The episode terminates with \textit{success} if the puck is in the range where hits can be made and the longitudinal speed is below the threshold.
    \item \textbf{Prepare}: The puck is initialized close to the table's boundary and is unsuitable for hitting. The task is to control the puck to move it into a good hit position. The puck is not allowed to cross the middle line.
    The episodes terminates if the puck crosses the middle line that connects the middle points of two goals, or the puck is on the opponent's side of the table.
    The episodes terminates with success if the puck is in the range where its can be made and the longitudinal speed is below the threshold.
\end{itemize}

\subsection{Tournament Stage}
Only the participant teams who submitted agents that were categorized as Deployable or Improvable were qualified to participate in the tournament stage.
The maximum number of teams is 16, which were determined based on the ranking of \textit{success rate} in the qualifying stage.
In this stage each team had to develop an agent able to play the whole game.
A hard coded baseline based on the implementation in \cite{baseline} agent was provided to test and validate the agents before the actual competition started. 

The tournament stage was divided into two rounds. After the first rounds teams had two weeks to adjust and improve their agent.
Teams were awarded three points if they won a match, otherwise one point if they drew and zero if they lost.
The final ranking of the tournament was determined by the total score of the two rounds.

\subsubsection{Game Rules}
\begin{itemize}
    \item A game is 15 minutes in length, i.e. 45000 steps.
    \item The first player to accumulate seven (7) points wins the game. If no player accumulates seven points within the designated competition time, the player with the most points wins.
    \item To score a point, the puck must fully enter the goal. Rebounds or pucks that get stuck halfway in do not count as a point.
    \item When a player makes a goal, the other player serves the puck next.
    \item A player may only strike the puck when it is on their side of the centerline.
    \item Mallets may not cross the center line when striking the puck.
    \item When the puck stops in the middle of the table (20cm width), the puck is reset randomly to one player's side
    \item “Topping” the puck is not allowed. This means players cannot lift their mallet and place it over the puck to hold it in place.
    \item Touching the puck with other parts of the robot is not allowed.
    \item Once the puck is on a certain player's side of the center line, the player has 15 seconds, i.e. 750 steps, to hit the puck back across the center line. 
    Otherwise a foul is committed and the opponent receives possession of the puck.
    \item If a player commits three fouls a point is given to the opponent.
\end{itemize}
